\documentclass[a4paper]{article}
\usepackage[utf8]{inputenc}
\usepackage[T1]{fontenc}
\usepackage{aeguill}
\usepackage{hyperref}
\usepackage{verbatim}
\usepackage{amsmath}
\usepackage{tikz}
\usepackage{a4wide}

\usepackage{stridman}

\newcommand{\strid}{\textsl{strid}}
\newcommand{\example}[1]{\begin{center}\fbox{\begin{minipage}{11cm}#1\end{minipage}}\end{center}}
\newcommand{\cinput}[1]{\vcenter{\hbox{\input{#1}}}}
\newcommand{\cpdfinput}[1]{\vcenter{\hbox{\includegraphics{#1}}}}

\title{strid -- A string diagrams generator}
\author{Samuel Mimram}

\hypersetup{
  pdftitle={\csname @title\endcsname},
  pdfauthor={\csname @author\endcsname},
%  unicode=true,
  colorlinks=true,
  linkcolor=black,
  citecolor=black,
  urlcolor=black
}

\begin{document}
\maketitle
\tableofcontents
\newpage

\strid{} is a string diagrams generator for inclusion into \LaTeX{} files.  It
is entirely programmed in OCaml\footnote{OCaml can be downloaded at
  \url{http://caml.inria.fr/}.}. Feel free to drop me a line at
\url{samuel.mimram@pps.jussieu.fr} if you have some comments, bug reports or
feature requests about it.

\section{Presentation of \strid{}}
\subsection{A first example}
%\begin{figure}[ht!]
\example{ Suppose that $(\mathcal{C},\otimes,I)$ is a strict monoidal
  category. A \emph{monoid} in $\mathcal{C}$ is an object $M$ of $\mathcal{C}$
  together with two maps $\mu:M\otimes M\to M$, called \emph{multiplication},
  and $\eta:I\to M$, called \emph{unit}, respectively drawn as
  \[
  \cpdfinput{monoid_mult.pdf}
  \]
  and
  \[
  \cpdfinput{monoid_unit.pdf}
  \]
  such that the equalities
  \[
  \cpdfinput{monoid_assoc_l.pdf}
  =
  \cpdfinput{monoid_assoc_r.pdf}
  \]
  and
  \[
  \cpdfinput{monoid_neutral_l.pdf}
  =
  \cpdfinput{monoid_neutral.pdf}
  =
  \cpdfinput{monoid_neutral_r.pdf}
  \]
  hold.
}
%\caption{Definition of monoid objects.}
%\end{figure}

Let's have a look at how we typeset the left member of the associativity
equation:
\[
\cpdfinput{monoid_assoc_l.pdf}
\]
The \strid{} code for this figure is
\verbatiminput{monoid_assoc_l.strid}

Despite it's apparent complexity, this code is very simple! Like every \strid{}
diagram, this code starts with ``\verb+matrix {+'' and ends with
  ``\verb+}+''. Between those lines comes the actual description of the
diagram. It is structured as a matrix whose colums are separated by ``\verb+&+''
and whose lines are separated by ``\verb+\\+''.

The rightmost multiplication is typeset by
\begin{center}
\verb+mult(ull,dl,r)+
\end{center}
Here, ``\verb+mult+'' is the kind of the operator (a multiplication-shaped one)
and its arguments specify that it should be linked to the relative positions
$(-2,1)$ (\verb+ull+ means up-left-left), $(-1,-1)$ (\verb+dl+ means down-left)
and $(1,0)$. The order in which the links should be specified is indicated on
the figure below:
\[
\cpdfinput{order_mult.pdf}
\]
As for other operators, links are specified inputs first and then outputs.

The labels are specified similarly by instructions like
\begin{center}
\verb+text(r)[l,t=#$M$#]+
\end{center}
This create a ``\verb+text+'' operator from here to the relative position
$(1,0)$. The brackets ``\verb+[l,t=#$M$#]+'' are here to specify optional
parameters related to this operator. The ``\verb+l+'' indicates that we are
going to add a label and the ``\verb+t=#$M$#+'' means that the label's text
should be ``\verb+$M$+''. The text between \verb+#+ is quoted uninterpreted.

Suppose that we have put the text of this figure in a file named
\verb+monoid_assoc_l.strid+. Compiling this file can be simply done by typing
\begin{center}
\verb+strid monoid_assoc_l.strid+
\end{center}
This generates a file \verb+monoid_assoc_l.tex+ which can be used in a \LaTeX{}
file like:
\begin{verbatim}
\documentclass{article}

\usepackage{tikz}

\begin{document}
\input{monoid_assoc_l.tex}
\end{document}
\end{verbatim}
You will need the \texttt{TikZ} package which can be downloaded at
\url{http://sourceforge.net/projects/pgf/}.

\bigskip

Similarly, the right member of the equation is generated in a file
\verb+monoid_assoc_r.tex+. To have the equality sign between the two diagrams
centered vertically you need to center the two diagrams. This can be done using
the \verb+\vcenter+ and \verb+\hbox+ \LaTeX{} commands as shown in the following
example:
\begin{verbatim}
\[
\vcenter{\hbox{\input{monoid_assoc_l.tex}}}
=
\vcenter{\hbox{\input{monoid_assoc_r.tex}}}
\]
\end{verbatim}

\subsection{Visualizing your diagram}
Making a nice diagram is sometimes hard and \LaTeX{} compilation of the diagrams
usually takes some time to complete. If you want to quickly see the diagram
generated by \strid{} on a file \texttt{toto.strid}, type the command
\begin{center}
  \texttt{strid -g toto.strid}
\end{center}
This will open a window in which the output diagram is displayed, which is
refreshed every time the file \texttt{toto.strid} is changed.


\subsection{Compiling}
Compiling diagrams can quickly become a tedious task : you have to run \strid{}
to generate a \LaTeX{} file and then use \texttt{latex} or \texttt{pdflatex} to
produce the final document (which can take long to compile). If you want to
speed up the compilation of the document, you can have \strid{} generate
directly pdf files by typing
\begin{center}
  \verb+strid --pdf toto.strid+
\end{center}
The resulting \texttt{toto.pdf} file can then be integrated in a \LaTeX{}
document as follows.
\begin{verbatim}
\documentclass{article}

\usepackage{graphics}

\begin{document}
\[
\includegraphics{toto.pdf}
\]
\end{document}
\end{verbatim}
In order to have the pdf files automatically generated from the strid files, a
\texttt{Makefile} file can be used, containing
\begin{verbatim}
STRIDFILES=$(wildcard *.strid)
STRIDPDF=$(STRIDFILES:.strid=.pdf)

pdf: $(STRIDPDF)

%.pdf: %.strid
        $(STRID) --pdf $<
\end{verbatim}
The pdf files corresponding to the strid files of the current directory can then
be generated by simply typing
\begin{center}
  \texttt{make}
\end{center}

Be careful, if you use macros in your strid files those won't be known when
generating the pdf. The solution is to put your macros in a \texttt{macros.sty}
file (or whatever name\texttt{.sty}) and include those in the tex file used to
produce the pdf by using the \verb+--latex-preamble+ command of \strid{}. For
example,
\[
\cpdfinput{add.pdf}
\]
is typeset in a file \texttt{add.strid} containing \verbatiminput{add.strid} and
is compiled with
\begin{center}
  \verb+strid --pdf --latex-preamble "\\usepackage{macros}" add.strid+
\end{center}
where the \texttt{macros.sty} file contains
\begin{verbatim}
\usepackage{amsfonts}
\newcommand{\N}{\mathbb{N}}
\end{verbatim}
The \texttt{Makefile} above can obviously be modified in order to cope with such
situations.

\section{The operators}
\subsection{Line: \texttt{line}}
\[
\cpdfinput{order_line.pdf}
\]
\subsection{Multiplication: \texttt{mult}}
\[
\cpdfinput{order_mult.pdf}
\]
\subsection{Unit: \texttt{unit}}
\[
\cpdfinput{order_unit.pdf}
\]
\subsection{Adjunction: \texttt{adj}}
\[
\cpdfinput{order_adj.pdf}
\]
\subsection{Symmetry: \texttt{sym}}
\[
\cpdfinput{order_sym.pdf}
\]
\subsection{Braiding: \texttt{braid}}
\[
\cpdfinput{order_braid.pdf}
\]
\subsection{$m,n$-ary box: \texttt{$m$box$n$}}
\[
\cpdfinput{order_nboxm.pdf}
\]
\subsection{Vertical box: \texttt{vbox}}
\[
\cpdfinput{order_vbox.pdf}
\]

\subsection{Region: \texttt{region}}
Regions can be delimited:
\[
\cpdfinput{pl_composition.pdf}
\]
is typeset by
\verbatiminput{pl_composition.strid}


\section{Parameters of operators}
\subsection{Labels}
Labels can be added to operators. For example the diagram
\[
\cpdfinput{mult_label.pdf}
\]
can be typeset by \verbatiminput{mult_label.strid} If you don't like the size of
the ellipse surrounding the label, this can of course be changed. For example,
\[
\cpdfinput{mult_small_label.pdf}
\]
can be typeset by
\verbatiminput{mult_small_label.strid}

Various shapes are available for labels:
\subsubsection{Triangles: \texttt{triangle} / \texttt{t}}
For example,
\[
\cpdfinput{bang_maybe_cut.pdf}
\]
can be typeset by
\verbatiminput{bang_maybe_cut.strid}

Here, the \texttt{s} parameter is the \emph{shape}, the \texttt{d} parameter is
the \emph{direction} of the triangle (here it is pointing down) and the
\texttt{c} parameter specifies the \emph{color} of the triangle. There are
shortcuts for the shapes, for example you can type \texttt{s=t} instead of
\texttt{s=triangle}, which is more concise but less readable.

There is a particular case where the direction of the triangle can be guessed
and you don't need to specify the direction of the triangle: when there is only
one output port. The \texttt{operad} operator does precisely this.
\[
\cpdfinput{operad.pdf}
\]
can be typeset by \verbatiminput{operad.strid} It also provides minor
improvements of the drawing for example, the output wire is exactly starting at
the vertex of the triangle :
\[
\cpdfinput{bang_maybe_cut_operad.pdf}
\]
is typeset by \verbatiminput{bang_maybe_cut_operad.strid}

\subsubsection{Rectangles: \texttt{rectangle} / \texttt{r}}
\[
\cpdfinput{monoidal_bifunct_l.pdf}
\quad=\quad
\cpdfinput{monoidal_bifunct_r.pdf}
\]
The left member is typeset by
\verbatiminput{monoidal_bifunct_l.strid}

\subsection{Arrows}
Lines can be oriented using the \texttt{a} attribute. For example,
\[
\cpdfinput{mult_a.pdf}
\]
can be typeset by \verbatiminput{mult_a.strid} To specify that the direction
should be backwards use the \texttt{d=b} subattribute. For example,
\[
\cpdfinput{mult_ab.pdf}
\]
can by typeset by \verbatiminput{mult_ab.strid} The position of the arrow can be
changed by setting the \texttt{t} parameter which is a float between \texttt{0.}
and \texttt{1.}. For example, the arrow can be put at the end of a line by using
the \texttt{t=1.} parameter :
\[
\cpdfinput{arrow.pdf}
\]
is typeset by
\verbatiminput{arrow.strid}


\section{Configuration files}
All parameters can be saved in a configuration file named
\texttt{strid.conf}. To generate a configuration file, type
\begin{verbatim}
strid --dump-conf
\end{verbatim}
You can then edit \texttt{strid.conf}.

Some of the options that can be set are:
\begin{itemize}
\item \verb+line_width+: default width of a line
\item \verb+label_width+: default width of a label
\item \verb+label_height+: default height of a label
\item \verb+no_tex_environment+: do not output \verb+\begin{tikz}+ and
    \verb+\end{tikz}+
\item \verb+scaling_factor+: scale the diagrams
\item \verb+label_triangle_height+: default height of a triangular label
\item \verb+label_rectangle_width+: default width of a rectangular label
\item \verb+label_rectangle_height+: default height of a rectangular label
\item \verb+interpolation+: interpolation method for drawing lines (possible
  values are \verb+cspline+ and \verb+linear+)
\item \verb+small_circle_ray+: ray of small circles (used to tweak the drawing
  of multiplications)
\end{itemize}


\section{Examples}
\subsection{Yang-Baxter equality for braids}
\[
\cpdfinput{yang_baxter_l.pdf}
\qquad=\qquad
\cpdfinput{yang_baxter_r.pdf}
\]
Left member is typeset by
\verbatiminput{yang_baxter_l.strid}
and right member by
\verbatiminput{yang_baxter_r.strid}

\subsection{Hopf law for bialgebras}
\[
\cpdfinput{bialgebra_hopf_l.pdf}
\quad=\quad
\cpdfinput{bialgebra_hopf_r.pdf}
\]
Left member is typeset by
\verbatiminput{bialgebra_hopf_l.strid}
and right member by
\verbatiminput{bialgebra_hopf_r.strid}

\subsection{Naturality condition for natural transformations between two lax functors between bicategories}
{\scriptsize
\[
\cinput{bicat_nat_transfo_l.tex}
\qquad
=
\qquad
\cinput{bicat_nat_transfo_r.tex}
\]
}

\noindent The code for the left-hand side of the equation is
{\scriptsize \verbatiminput{bicat_nat_transfo_l.strid}}
\noindent and the code for the right-hand side is
{\scriptsize \verbatiminput{bicat_nat_transfo_r.strid}}

\end{document}
