\documentclass{article}
\usepackage[utf8]{inputenc}
\usepackage[T1]{fontenc}
\usepackage{aeguill}
\usepackage{amsmath}
\usepackage{tikz}

\newcommand{\strid}{\textsl{strid}}
\newcommand{\example}[1]{\begin{center}\fbox{\begin{minipage}{11cm}#1\end{minipage}}\end{center}}

\title{strid -- A string diagrams generator}
\author{Samuel Mimram}
\begin{document}
\maketitle

\strid{} is a string diagrams generator for inclusion into \LaTeX{} files.

\section{A first example}
\example{
Suppose that $(\mathcal{C},\otimes,I)$ is a strict monoidal category. A \emph{monoid} in $\mathcal{C}$ is an object $M$ of $\mathcal{C}$ together with two maps $\mu:M\otimes M\to M$, called \emph{multiplication}, and $\eta:I\to M$, called \emph{unit}, respectively drawn as
\[
\input{monoid_mult.tex}
\]
and
\[
\input{monoid_unit.tex}
\]
such that the equalities
\[
\input{monoid_assoc_l.tex}
\quad=\quad
\input{monoid_assoc_r.tex}
\]
and
\[
\input{monoid_neutral_l.tex}
\quad=\quad
\input{monoid_neutral.tex}
\quad=\quad
\input{monoid_neutral_r.tex}
\]
hold.
}

Let's have a look at how we typeset the left member of the associativity equation:
\[
\input{monoid_assoc_l.tex}
\]
The \strid{} code for this figure is
\begin{verbatim}
matrix {
text(,r)[l,t=#$M$#]\\
text(,r)[l,t=#$M$#]&&&mult(ull,dl,r)&text(l,)[l,t=#$M$#]\\
&&mult(ul,dl,)\\
text(,r)[l,t=#$M$#]\\
}
\end{verbatim}
Despite it's apparent complexity, this code is very simple! Like every \strid{} diagram, this code starts with ``\verb+matrix {+'' and ends with ``\verb+}+''. Between those lines comes the actual description of the diagram. It is structured as a matrix whose colums are separated by ``\verb+&+'' and whose lines are separated by ``\verb+\\+''.
\end{document}
